\chapter{Introduction}

I'll be honest, I am not entirely sure how to write this, so I'll start with a disclaimer: this work is not meant to spark any political discussion.
It \textit{is} meant to discuss how we, as a society, engage with the research literature and is split up into 4 distinct strategies:

\begin{enumerate}
\item Rejection of science
\item Deification of researchers
\item Research with immediate gratification
\item Research without immediate gratification
\end{enumerate}

Each of these ideas will be discussed within the context of a greater, fictional universe.
I recognize that each of these topics can be stretched into theses in their own right, but my PhD is in computational science, not sociology.
More than that, I recognize that the general public -- those I am trying to reach the most with this work -- do not read scientific literature.
For that reason, I chose to formulate my ideas not as a scientific hypothesis, but as an opinion stated in fiction.

I believe that statement requires re-emphasis: any ideas expressed within this work are purely my own opinions and ideas.
If you agree with them, great!
If you disagree, also great!
I want this work to spark discussion, but I don't want it to cause any sort of division.

Unfortunately, here is where I am a little split.
The author in me wants to invoke emotion.
The researcher in me wants to be as accurate as possible.
As such, I might provide additional clarification where needed, such as academic citations or a some back-of-the-envelope calculations used to justify events in-universe\footnote{These statements will usually come in the form of footnotes, like these, but might also require separate figures or example}.
If you are interested in the narrative only, these statements can be safely ignored.

Finally, I will try to keep this novel available for free online\footnote{at https://github.com/leios/the\_drift} under a Creative Commons License\footnote{In particular, a Creative Commons Attribution, Non-Commercial, Share-Alike (CC-BY-NC-SA) license. This means that anyone is free to use this work for anything they like, so long as it is not for a commercial purpose and any derivative works have the same license. If people use this work, they must attribute me, Dr. James Schloss (Leios)}.
If possible, I would like to have discussion about this work available on github for everyone to see and take part in.
In fact, because of my licensing and openness, I am not going to try to publish this through traditional means; therefore, if you notice any typos, loose narrative threads, poor writing, etc, feel free to create an issue (or Pull Request online).
Again, I encourage any and all engagement with this work.
Also: because this work is freely available online, please do not pay for this unless you:
\begin{enumerate}
\item Want to support me, as a creator
\item Want a physical (or ebook) copy
\item Are buying this for someone else as a gift\footnote{To be honest, I don't know why you would do that, but thanks for the support!}
\end{enumerate}

Ok, that's it for the foreward.
I know most people skip this part of the book entirely and instead opt to hop right into the content, but due to the nonconventional nature of this work, I thought it was appropriate to discuss everything ahead-of-time.

